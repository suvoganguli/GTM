
% This LaTeX was auto-generated from an M-file by MATLAB.
% To make changes, update the M-file and republish this document.

\documentclass{article}
\usepackage{graphicx}
\usepackage{color}
\setcounter{tocdepth}{2}
\sloppy
\definecolor{lightgray}{gray}{0.5}
\setlength{\parindent}{0pt}
\textwidth       6.25in
\textheight      8.75in
\footskip        0.5in
\oddsidemargin  -.25in
\evensidemargin -.25in
\topmargin      -0.5in
\begin{document}

\vspace{.5in}
\centerline{\bf \Large gtm\_design: Mfile Documentation} 
\vspace{1.0in}
\tableofcontents

\newpage

\section*{function appendmws(MWS,model)}
\addcontentsline{toc}{section}{appendmws(MWS,model)}
 \begin{verbatim}
  Appends model workspace with variables 
  given by the fields of the structure MWS.  
 
  ! Warning !  Will overwrite indentically named variables without warning!
 
  Inputs:
   MWS   - Model Workspace Structure, contains simulation parameters
   model - Name of model to load into, default is 'gtm_design'
\end{verbatim}
\newpage

\section*{function clearmws(model);}
\addcontentsline{toc}{section}{clearmws(model);}
\begin{verbatim}
  Clears the current model workspace.
   
  Inputs:
     model   - name of model, defaults to 'gtm_design'
   
\end{verbatim}
\newpage

\section*{function DCM = euler321(p);}
\addcontentsline{toc}{section}{DCM = euler321(p);}
\begin{verbatim} 
  Creates directional cosine matrix for euler 321 rotation 
  sequence. 
  
  Inputs:
    p    - phi/theta/psi vector, in radians
  
  Outputs:
    DCM  - Directional cosine matrix
  	
\end{verbatim}
\newpage

\section*{function expand(structure);}
\addcontentsline{toc}{section}{expand(structure);}
\begin{verbatim} 
  Given a data structure this expands each field into
  a separate named variable in the callers workspace
\end{verbatim}
\newpage

\section*{function MWS = grabmws(model);}
\addcontentsline{toc}{section}{MWS = grabmws(model);}
\begin{verbatim} 
  Grabs current model workspace and returns as
  fields in a strucure
  
  Inputs:
     model  - name of model, defaults to 'gtm_design'
  
  Outputs:
     MWS    - simlation parameters from the model workspace
  
\end{verbatim}
\newpage

\section*{function MWS = init\_design(vehicle);}
\addcontentsline{toc}{section}{MWS = init\_design(vehicle);}
\begin{verbatim} 
  Creates Model Workspace Structure for gtm_design simulation.
 
  Inputs:
    vehicle     - type of vehicle, either 'GTM' or 'S2' (default = 'GTM')
    sensors_on  - flag for turning sensor models on  (default = 0)
    noise_on    - flag for turning sensor noise models on  (default = 0)
    fuelburn_on - flag for turning fuel burn model on  (default = 0)
 
  Outputs:
    MWS     - Simulation parameters, to be loaded into Model Workspace
              with loadmws(MWS) or appendmws(MWS).
 
\end{verbatim}
\newpage

\section*{function [sys,londyn,latdyn] = linmodel(MWS,vabflag,SplitSurf,Ts)}
\addcontentsline{toc}{section}{[sys,londyn,latdyn] = linmodel(MWS,vabflag,SplitSurf,Ts)}
\begin{verbatim} 
 Linearize gtm_design simulation at current trim point

 Inputs:
  MWS     -  simulation parameters, defaults to pre-loaded model workspace
  vabflag  - flag to get linear models in terms of V, alpha, beta (0 default)
  SplitSurf- flag for returning independent control surface inputs rather
             than grouped surfaces (0 default)
  Ts       - Sample time, zero for continuous model (0 default)

 Outputs:
   sys    - 6dof system with control surface inputs/state outputs
   londyn - Approximate (4th order) longitudinal dynamics
   latdyn - Approximate (4th order) lateral dynamics

\end{verbatim}
\newpage

\section*{function loadmws(MWS,model);}
\addcontentsline{toc}{section}{loadmws(MWS,model);}
\begin{verbatim} 
  Clears model workspace and replaces with variables 
  given by the fields of the structure MWS
 
  Inputs:
   MWS   - Model Workspace Structure, contains simulation parameters
   model - Name of model to load into, default is 'gtm_design'
 
\end{verbatim}
\newpage

\section*{function MWS\_out = seteomic(MWS,statename,value,...)}
\addcontentsline{toc}{section}{MWS\_out = seteomic(MWS,statename,value,...)}
\begin{verbatim}
  This function allows the initial condtions of the gtm_design simulation's 
  equation of motion (MWS.StatesInp) to be specified by named values. 
 
  Specifically statename is one of:
      tas   - total airspeed (knots)
      alpha - angle of attack (deg)
      beta  - sideslip (deg)
      p     - body rate (deg/sec)
      q     - body rate (deg/sec)
      r     - body rate (deg/sec)
      lat   - latitude (deg)
      lon   - longitude (deg)
      alt   - altitude (ft)
      phi   - Euler angle (deg)
      theta - Euler angle (deg)
      psi   - Euler angle (deg)
 
  Values not explictly specified remain at the value they had in MWS
 
  Examples: 
      MWS=seteomic(init_design(),'alt',3000);
      MWS=seteomic(MWS,'tas',90,'beta',3,'p',15);  
 
\end{verbatim}
\newpage

\section*{function [MWS,Xtrim,Trimcond,Err] = trimgtm(target,PitchSurf,verbose);}
\addcontentsline{toc}{section}{[MWS,Xtrim,Trimcond,Err] = trimgtm(target,PitchSurf,verbose);}
\begin{verbatim} 
  Trims gtm_design simulation to target conditions.
 
  Inputs:
   Target is a structure which has some subset of the following fields:
      eas          - Equivalent air speed (knots)
      tas          - True airspeed (knots)
      alpha        - Angle of attack (deg)
      beta         - Sideslip (deg), defaults to zero
      gamma        - Flight path angle (deg)
      roll         - Roll angle (deg)
      pitch        - Pitch angle (deg)
      yaw          - Heading angle (0-360)
      rollrate     - d/dt(phi) (deg/sec), defaults to zero
      pitchrate    - d/dt(theta) (deg/sec), defaults to zero
      yawrate      - d/dt(psi) (deg/sec), defaults to zero
      pdeg         - Angular velocity (deg/sec)
      qdeg         - Angular velocity (deg/sec)
      rdeg         - Angular velocity (deg/sec)
 
    PitchSurf    - 'elev' for elevator (default) or 'stab' for stabalizer
    verbose      - 0(quiet), 1(soln disp), or 2(iterations), defaults to 2
 
 Outputs:
    MWS    - simulation parameter set, updated to be at trim condition
    Xtrim    - simulation full-state, at trim
    Trimcond - values of state and control surfaces at trim.
    Err      - max error in trim condition
 
  Unspecified variables are free, or defaulted to the values shown above.
  To force a defaulted variable to be free define it with an empty matrix.
  For example, by default beta=0  but "target.beta=[];" will allow beta
  to be free in searching for a trim condition.
 
  Examples:
   MWS=trimAirSTAR(struct('eas',95,'gamma',0));
   [MWS,Xt,Tc,Er]=trimAirSTAR(struct('alpha',3,'gamma',0,'yaw',120),1);
   [MWS,Xt,Tc,Er]=trimAirSTAR(struct('tas',100,'gamma',3,'yawrate',10),0,0);
\end{verbatim}



\end{document}
    
